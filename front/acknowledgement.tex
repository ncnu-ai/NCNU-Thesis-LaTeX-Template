% !TeX root = ../main.tex

\begin{acknowledgement}

常到外國朋友家吃飯。當蠟燭燃起,菜肴布好,客主就位,總是主人家的小男孩或小女孩舉起小手,低頭感謝上天的賜予,並歡迎客人的到來。

我剛到美國時,常鬧得尷尬。因為在國內養成的習慣,還沒有坐好,就開動了。

以後凡到朋友家吃飯時,總是先囑咐自己;今天不要忘了,可別太快開動啊!幾年來,我已變得很習慣了。但我一直認為只是一種不同的風俗儀式,在我這方面看來,忘或不忘,也沒有太大的關係。

前年有一次,我又是到一家去吃飯。而這次卻是由主人家的祖母謝飯。她雪白的頭髮,顫抖的聲音,在搖曳的燭光下,使我想起兒時的祖母。那天晚上,我忽然覺得我平靜如水的情感翻起滔天巨浪來。

在小時候,每當冬夜,我們一大家人圍著個大圓桌吃飯。我總是坐在祖母身旁。祖母總是摸著我的頭說:「老天爺賞我們家飽飯吃,記住,飯碗裡一粒米都不許剩,要是蹧蹋糧食,老天爺就不給咱們飯了。」

剛上小學的我,正在念打倒偶像及破除迷信等為內容的課文,我的學校就是從前的關帝廟,我的書桌就是供桌,我曾給周倉畫上眼鏡,給關平戴上鬍子,祖母的話,老天爺也者,我覺得是既多餘,又落伍的。

不過,我卻很尊敬我的祖父母,因為這飯確實是他們掙的,這家確實是他們立的。我感謝面前的祖父母,不必感謝渺茫的老天爺。

\end{acknowledgement}